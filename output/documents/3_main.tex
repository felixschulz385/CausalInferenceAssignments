\documentclass{scrartcl}

\usepackage[utf8x]{inputenc}
\usepackage{graphicx}
\usepackage{booktabs}
\usepackage{titling}
\usepackage{xcolor}
\usepackage{amsmath}
\usepackage{amsfonts}
\usepackage[a4paper, margin=.9in]{geometry}

\setlength{\droptitle}{-7.5em}

\title{Causal Inference for Policy Evaluation\\
\Large{Assignment 3}}
\author{Marco Gortan, Felix Schulz, Benjamin Weggelaar}
\date{\today}

\newcommand{\marco}[1]{\textcolor{red}{#1}}
\newcommand{\felix}[1]{\textcolor{cyan}{#1}}
\newcommand{\benji}[1]{\textcolor{green}{#1}}

\begin{document}

\maketitle

\section*{Instrumental Variables}

\subsection*{Question 1}

\paragraph*{(a)}
% How could what you observe affect the intra-household division of (market) labour supply, domestic chores and childcare?


% Table created by stargazer v.5.2.3 by Marek Hlavac, Social Policy Institute. E-mail: marek.hlavac at gmail.com
% Date and time: Mon, Apr 28, 2025 - 10:40:24
% Requires LaTeX packages: dcolumn 
\begin{table}[!htbp] \centering 
  \caption{Parental Characteristics} 
  \label{tab:avg_age} 
\begin{tabular}{@{\extracolsep{5pt}} D{.}{.}{-2} D{.}{.}{-2} D{.}{.}{-2} D{.}{.}{-2} } 
\\[-1.8ex]\hline 
\hline \\[-1.8ex] 
\multicolumn{1}{c}{Average age at First Birth (Mother)} & \multicolumn{1}{c}{Average age at First Birth (Father)} & \multicolumn{1}{c}{Average income (Mother)} & \multicolumn{1}{c}{Average income (Father)} \\ 
\hline \\[-1.8ex] 
20.90 & 24.03 & 6,232.10 & 39,030.66 \\ 
\hline \\[-1.8ex] 
\multicolumn{4}{l}{Age at first birth and income of parents in the sample of married couples.} \\ 
\end{tabular} 
\end{table} 



\begin{tabular}{l D{.}{.}{5.5} D{.}{.}{5.5}}
\toprule
 & \multicolumn{1}{c}{2(b)} & \multicolumn{1}{c}{2(c)} \\
\midrule
(Intercept) & 6.17^{***}  & 4.35^{***}  \\
            & (0.88)      & (1.24)      \\
morekids    & -7.81^{***} & -2.28       \\
            & (0.16)      & (2.65)      \\
agem        & 1.41^{***}  & 1.22^{***}  \\
            & (0.03)      & (0.09)      \\
agefstm     & -1.31^{***} & -1.06^{***} \\
            & (0.04)      & (0.13)      \\
blackm      & 11.14^{***} & 10.79^{***} \\
            & (0.36)      & (0.40)      \\
hispm       & 1.54^{***}  & 0.80        \\
            & (0.44)      & (0.57)      \\
othracem    & 3.13^{***}  & 2.85^{***}  \\
            & (0.49)      & (0.52)      \\
\midrule
R$^2$       & 0.06        & 0.05        \\
Adj. R$^2$  & 0.06        & 0.05        \\
Num. obs.   & 69299       & 69299       \\
\bottomrule
\multicolumn{3}{l}{\scriptsize{$^{***}p<0.001$; $^{**}p<0.01$; $^{*}p<0.05$}}
\end{tabular}


\paragraph*{(d)}
% Compare methods, 2SLS estimates, first stages.

Weirdly enough, all estimates are not significant?? is this what the exercise asks us to do?

\paragraph*{(e)}
% Compare the OLS estimate to the 2SLS you got before and also the respective standard errors. What do you conclude? Which one would you use for inference?

\subsection*{Question 2}

\paragraph*{(a)}

\begin{table}

\caption{\label{tab:tab:children_age_group}Share with \textgreater 2 Children by Age Group}
\centering
\begin{tabular}[t]{ccc}
\toprule
Age Group & Share with \textgreater 2 Children & N\\
\midrule
$age > 31$ & 0.42 & 56426\\
$age \leq 31$ & 0.34 & 69299\\
\bottomrule
\end{tabular}
\end{table}


\paragraph*{(b)}
% Why would you control specifically for these variables? Are these control variables sufficient to alleviate potential endogeneity concerns? Why yes or why not? 

\paragraph*{(c)}
% Compare your results to the OLS estimates and comment. 


\begin{tabular}{l D{.}{.}{5.5} D{.}{.}{5.5}}
\toprule
 & \multicolumn{1}{c}{2(b)} & \multicolumn{1}{c}{2(c)} \\
\midrule
(Intercept) & 6.17^{***}  & 4.35^{***}  \\
            & (0.88)      & (1.24)      \\
morekids    & -7.81^{***} & -2.28       \\
            & (0.16)      & (2.65)      \\
agem        & 1.41^{***}  & 1.22^{***}  \\
            & (0.03)      & (0.09)      \\
agefstm     & -1.31^{***} & -1.06^{***} \\
            & (0.04)      & (0.13)      \\
blackm      & 11.14^{***} & 10.79^{***} \\
            & (0.36)      & (0.40)      \\
hispm       & 1.54^{***}  & 0.80        \\
            & (0.44)      & (0.57)      \\
othracem    & 3.13^{***}  & 2.85^{***}  \\
            & (0.49)      & (0.52)      \\
\midrule
R$^2$       & 0.06        & 0.05        \\
Adj. R$^2$  & 0.06        & 0.05        \\
Num. obs.   & 69299       & 69299       \\
\bottomrule
\multicolumn{3}{l}{\scriptsize{$^{***}p<0.001$; $^{**}p<0.01$; $^{*}p<0.05$}}
\end{tabular}


\paragraph*{(d)}
% Why might the sample restrictions you imposed (women below the median age) be problematic for the 2SLS approach? 

\section*{Question 3}

\paragraph*{(a)}
% (Task description)
% Independent of the instrument used, Angrist and Evans (1998) find no economically and statistically
% significant effect on male labour supply. Check the employment rates in AngristEvans1980_reduced.RData
% for men and women, and how common part-time work is among employed men and women.
% Compare the results with the rates for Switzerland outlined above.
% Based on that, discuss how you would expect the effect of fertility on male labour supply
% in Switzerland today to differ from those in Angrist and Evans (1998).

\paragraph*{(b)}
% (Task description)
% Discuss how you would use a 2SLS approach to estimate the effect of fertility
% on labour supply in Switzerland today.

\paragraph*{(c)}
% (Task description)
% Discuss how you would modify the sample restrictions that we imposed
% in the IV lab session on the data if we were to use recent data from Switzerland.

\subsection*{Question 4 (PhD Only)}

\paragraph*{(a)}
% (Task description)
% Which test can be performed to test for the exogeneity of the instruments in a parametric IV?
% Explain the logic behind it, and clearly explain what hypothesis we can test.

\paragraph*{(b)}
% (Task description)
% Under which assumptions will the test be informative about instrument validity?

\section*{Regression Discontinuity Design}

\subsection*{Question 6}

\paragraph*{(a)}
% (Task description)
% In addition to data from the 2000 population census,
% the author uses information on outcomes in 1990, which is before the assignment to the treatment.
% What does he use it for and why is it in support of the identification strategy?

\paragraph*{(b)}
% (Task description)
% The author determines the optimal bandwidth using the algorithm developed by
% Imbens and Kalyanaraman (2012).
% Briefly explain why it yields a smaller optimal bandwidth for women.

\paragraph*{(c)}
% (Task description)
% For women, the author proposes and tests a mechanism that can explain the positive RD estimate
% (despite a negative raw correlation).
% Explain this mechanism and comment on the test procedures.

\end{document}